\documentclass[
11pt, % The default document font size, options: 10pt, 11pt, 12pt
codirector, % Uncomment to add a codirector to the title page
]{charter} 




% El títulos de la memoria, se usa en la carátula y se puede usar el cualquier lugar del documento con el comando \ttitle
\titulo{Sistema de monitoreo inteligente de consumo de agua} 

% Nombre del posgrado, se usa en la carátula y se puede usar el cualquier lugar del documento con el comando \degreename
%\posgrado{Carrera de Especialización en Sistemas Embebidos} 
\posgrado{Carrera de Especialización en Internet de las Cosas} 
%\posgrado{Carrera de Especialización en Intelegencia Artificial}
%\posgrado{Maestría en Sistemas Embebidos} 
%\posgrado{Maestría en Internet de las cosas}

% Tu nombre, se puede usar el cualquier lugar del documento con el comando \authorname
\autor{Ing. Paolo Gonzalo Bazán Hernández} 

% El nombre del director y co-director, se puede usar el cualquier lugar del documento con el comando \supname y \cosupname y \pertesupname y \pertecosupname
\director{Por definir}
\pertenenciaDirector{FIUBA} 
% FIXME:NO IMPLEMENTADO EL CODIRECTOR ni su pertenencia
\codirector{Por definir} % para que aparezca en la portada se debe descomentar la opción codirector en el documentclass
\pertenenciaCoDirector{FIUBA}

% Nombre del cliente, quien va a aprobar los resultados del proyecto, se puede usar con el comando \clientename y \empclientename
\cliente{Marleny López Machuca}
\empresaCliente{Cliente particular}

% Nombre y pertenencia de los jurados, se pueden usar el cualquier lugar del documento con el comando \jurunoname, \jurdosname y \jurtresname y \perteunoname, \pertedosname y \pertetresname.
\juradoUno{Nombre y Apellido (1)}
\pertenenciaJurUno{pertenencia (1)} 
\juradoDos{Nombre y Apellido (2)}
\pertenenciaJurDos{pertenencia (2)}
\juradoTres{Nombre y Apellido (3)}
\pertenenciaJurTres{pertenencia (3)}
 
\fechaINICIO{21 de octubre de 2021}		%Fecha de inicio de la cursada de GdP \fechaInicioName
\fechaFINALPlan{09 de diciembre de 2021} 	%Fecha de final de cursada de GdP
\fechaFINALTrabajo{15 de mayo de 2022}	%Fecha de defensa pública del trabajo final


\begin{document}

\maketitle
\thispagestyle{empty}
\pagebreak


\thispagestyle{empty}
{\setlength{\parskip}{0pt}
\tableofcontents{}
}
\pagebreak


\section*{Registros de cambios}
\label{sec:registro}


\begin{table}[ht]
\label{tab:registro}
\centering
\begin{tabularx}{\linewidth}{@{}|c|X|c|@{}}
\hline
\rowcolor[HTML]{C0C0C0} 
Revisión & \multicolumn{1}{c|}{\cellcolor[HTML]{C0C0C0}Detalles de los cambios realizados} & Fecha      \\ \hline
0      & Creación del documento                                 & 21/10/2021 \\ \hline %\fechaInicioName \\ \hline
1      & Se completa hasta el punto 1 inclusive                 & 04/11/2021 \\ \hline
2      & Se completa hasta el punto 9 inclusive y correcciones de rev.1	& 11/11/2021 \\ \hline
%		  Se puede agregar algo más \newline
%		  En distintas líneas \newline
%		  Así                                                    & dd/mm/aaaa \\ \hline
%3      & Se completa hasta el punto 11 inclusive                & dd/mm/aaaa \\ \hline
%4      & Se completa el plan	                                 & dd/mm/aaaa \\ \hline
\end{tabularx}
\end{table}

\pagebreak



\section*{Acta de constitución del proyecto}
\label{sec:acta}

\begin{flushright}
Buenos Aires, \fechaInicioName
\end{flushright}

\vspace{2cm}

Por medio de la presente se acuerda con el \authorname\hspace{1px} que su Trabajo Final de la \degreename\hspace{1px} se titulará ``\ttitle''. Consistirá en la implementación de un prototipo de un sistema de control del flujo de agua de un punto de salida, y tendrá un presupuesto preliminar estimado de 600 hs de trabajo y USD8200, con fecha de inicio \fechaInicioName\hspace{1px} y fecha de presentación pública \fechaFinalName.

Se adjunta a esta acta la planificación inicial.

\vfill

% Esta parte se construye sola con la información que hayan cargado en el preámbulo del documento y no debe modificarla
\begin{table}[ht]
\centering
\begin{tabular}{ccc}
\begin{tabular}[c]{@{}c@{}}Ariel Lutenberg \\ Director posgrado FIUBA\end{tabular} & \hspace{2cm} & \begin{tabular}[c]{@{}c@{}}\clientename \\ \empclientename \end{tabular} \vspace{2.5cm} \\ 
\multicolumn{3}{c}{\begin{tabular}[c]{@{}c@{}} \supname \\ Director del Trabajo Final\end{tabular}} \vspace{2.5cm} \\
%\begin{tabular}[c]{@{}c@{}}\jurunoname \\ Jurado del Trabajo Final\end{tabular}     &  & \begin{tabular}[c]{@{}c@{}}\jurdosname\\ Jurado del Trabajo Final\end{tabular}  \vspace{2.5cm}  \\
%\multicolumn{3}{c}{\begin{tabular}[c]{@{}c@{}} \jurtresname\\ Jurado del Trabajo Final\end{tabular}} \vspace{.5cm}                                                                     
\end{tabular}
\end{table}




\section{1. Descripción técnica-conceptual del proyecto a realizar}
\label{sec:descripcion}


\begin{consigna}{black} % El bloque "consigna" se usa para poner texto en rojo y dar una pequeña ayuda sobre cómo completar la sección

\textbf{Situación problemática} 

En un estudio llevado a cambio por el Centro de Resiliencia de Estocolmo publicado en el 2009, se identificó que el planeta Tierra tiene nueve límites, procesos o parámetros interconectados que son determinantes para mantener la estabilidad del planeta. De cruzar estos límites, se afectará el equilibrio vital con consecuencias y cambios irreversibles que pueden desencadenar el colapso de nuestra sociedad.

Uno de estos parámetros, es el uso del agua dulce. Si bien la Tierra tiene mucha agua, la gran parte es salada y solo el 2.5\% es dulce. Este porcentaje es cada vez menor por el cada vez mayor uso de la agricultura, que representa el 70\% del total de agua dulce. Esta tasa va incrementándose anualmente por el crecimiento poblacional. Es indispensable promover una cultura de ahorro de este líquido vital.

En las ciudades, con la complejidad de las redes urbanas de agua potable y la multiplicación de puntos de agua, también se incrementan las fugas por agrietamiento o rotura de las tuberías. Ante eventos de esta naturaleza, dependiendo de su magnitud de la fuga, si se tratara por ejemplo, de la rotura de la tubería alimentadora de una casa, la inundación será advertida rápidamente por sus residentes. Sin embargo, si las fugas fueran ligeras, será difícil tener una detección temprana. Por lo tanto, las acciones correctivas tomarán tiempo en ejecutarse.

Para un mejor entendimiento, en el siguiente gráfico se puede identificar la cantidad de agua que se podría desperdiciar en un hogar que tuviera un solo punto de agua defectuoso, que desperdicie 30 gotas por minuto. Este ritmo podría pasar inadvertido por el propietario, pero sí logrará que se arroje alrededor de 1041 galones de agua al año o su equivalente de 3650 litros.

\begin{figure}[htpb]
\centering 
\includegraphics[width=.8\textwidth]{./Figuras/consumoAgua.png}
\caption{Cálculo estimado de pérdida de agua en un hogar}
\label{fig:consumoAgua}
\end{figure}

La necesidad principal a satisfacer en el mercado es la carencia de información en tiempo real que permita la rápida atención de fugas y desperdicio del agua potable en los hogares u organizaciones.


\textbf{Descripción técnica/conceptual del proyecto a realizar} 

El proyecto busca construir un sistema que permita el control del consumo de agua potable y la detección de fugas en redes caseras o empresariales de agua potable. 

El trabajo propuesto se abordará con un sensor inalámbrico de caudal de agua que permita la lectura del caudal que pasa por el punto monitoreado. Los datos deberán ser enviados a través del protocolo MQTT para su almacenamiento en una base de datos. En el frente de visualización de la información, se propone la creación de una aplicación que permita mostrar el flujo de agua en el punto monitoreado, así como su consumo acumulado durante un tiempo definido. 

El principal desafío del presente proyecto es la identificación del sensor que permita la correcta medición del flujo de agua. En el mercado existe disponibilidad de contadores de caudal de agua. Sin embargo, se trata de sensores intrusivos, ya que requieren alterar la plomería para incorporarlos. Se busca que el proyecto no sea intrusivo para una fácil utilización. Para lograr este objetivo, se necesitará usar dispositivos que utilicen la técnica de ultrasonido, que si bien existen en el mercado, no lo hace de forma masiva.

Los contadores inteligentes de caudal de agua poseen tecnología madura que las empresas que gestionan el agua potable en las ciudades, como AySA, irán desplegando paulatinamente. A mediano/largo plazo, estos dispositivos enviarán la información en tiempo real a sus sistemas centrales y eventualmente, se compartirá con los usuarios finales.

La motivación del presente proyecto es poder brindar al mercado una herramienta económica para la medición de caudal en tiempo real. Esta solución podrá estar disponible para hogares, organizaciones o para quienes brinden servicios relacionados. Se podrá conocer de forma temprana el consumo actual y detectar, basándose en comparación de consumos anteriores, fugas de agua potable.

En la Figura \ref{fig:diagHLSolucion} se presenta el diagrama a alto nivel de la solución.


%\vspace{25px}

\begin{figure}[htpb]
\centering 
\includegraphics[width=.8\textwidth]{./Figuras/diagHLSolucion.png} 
\caption{Diagrama a alto nivel de la solución}
\label{fig:diagHLSolucion}
\end{figure}

\vspace{25px}

\end{consigna}


\section{2. Identificación y análisis de los interesados}
\label{sec:interesados}

\begin{consigna}{black} 

\begin{table}[ht]
%\caption{Identificación de los interesados}
%\label{tab:interesados}
\begin{tabularx}{\linewidth}{@{}|l|X|X|l|@{}}
\hline
\rowcolor[HTML]{C0C0C0} 

Rol           & Nombre y Apellido & Organización 	& Puesto 	\\ \hline
Cliente       & Marleny López Machuca & Cliente particular	& -- 	\\ \hline
Responsable   & \authorname       & FIUBA 			& Alumno 	\\ \hline
Orientador    & Por definir       & Por definir  	& Director  \\ \hline %\pertesupname
Colaborador   & Por definir       & Por definir     & Co-director \\ \hline

\end{tabularx}
\end{table}

 
\begin{itemize}
	\item \textbf{Orientador:} Posee conocimiento y experiencia en soluciones TI que ayudará en la estrategia y definición de los objetivos del proyecto. Cuenta con escaso tiempo disponible. Se deberán planificar reuniones con anticipación de 15 días.
	\item \textbf{Colaborador:} Posee conocimiento y experiencia en soluciones IoT que ayudará en los conceptos y aspectos técnicos del proyecto. Cuenta con escaso tiempo disponible. Se deberán planificar reuniones con anticipación de 15 días.
\end{itemize}

\end{consigna}



\section{3. Propósito del proyecto}
\label{sec:proposito}

\begin{consigna}{black}
El propósito de este proyecto es mejorar el conocimiento sobre sensorización de caudal de fluidos, específicamente de agua. Se busca ampliar el conocimiento en la integración de los componentes de medición, protocolos de comunicación y servicios de software que pongan en valor la información sobre el consumo de agua potable.
\end{consigna}

\section{4. Alcance del proyecto}
\label{sec:alcance}

\begin{consigna}{black}

El proyecto de trabajo final del CEIoT tendrá como alcance lo siguiente:

\begin{enumerate}
\item Diseño, identificación e implementación de prototipo de solución de telemetría, para medición de caudal de agua en una tubería residencial
\begin{itemize}
	\item Basada en utilización de sensores de medición de caudal de agua con tecnología ultrasónica
	\item Los sensores no serán instalados "en línea" en la tubería o cañería. Serán instalados de forma no intrusiva, sin intervenirla.
	\item La comunicación hacia la solución de software será inalámbrica, utilizando preferentemente protocolo Wi-Fi
\end{itemize}

\item Diseño y construcción de solución de software para la captura y procesamiento de datos
\begin{itemize}
	\item Basada en tecnología de contenedores
	\item Utilización de base de datos relacional
\end{itemize}

\item Diseño y construcción de solución de software front-end para la presentación de datos en tiempo real y de forma histórica
\begin{itemize}
	\item Basado en lenguaje typescript
\end{itemize}

\end{enumerate}

El presente proyecto no incluye:

\begin{itemize}
	\item Detección de caudal de otro tipo de fluidos
	\item Inclusión de servicios de nube
	\item Gestión de analítica de datos
\end{itemize}


\end{consigna}


\section{5. Supuestos del proyecto}
\label{sec:supuestos}

\begin{consigna}{black}
Para el desarrollo del presente proyecto se supone que: 

\begin{itemize}
	\item Se dispondrá de tiempo suficiente para el cumplimiento de los objetivos y entregables
	\item Se dispondrá de recursos económicos para la adquición de materiales necesarios para la construcción del prototipo y software asociado
	\item Se dispondrá de orientación oportuna por parte de todos los involucrados de acuerdo a su rol
	\item Se asume niveles de precisión aceptables en la tecnología de los dispositivos de telemetría elegida
\end{itemize}

\end{consigna}

\section{6. Requerimientos}
\label{sec:requerimientos}

\begin{consigna}{black}
Los requerimientos del proyecto serán listados de acuerdo al siguiente criterio:
\begin{itemize}
	\item Su clasificación será: funcionales y no funcionales.
	\item La prioridad de los requerimientos se listan de mayor a menor.
	\item El cumplimiento de estos requerimientos es mandatorio, salvo se especifique si fuera opcional.
\end{itemize}

\begin{enumerate}
\item Requerimientos Funcionales

	\begin{enumerate}
		\item Conexión de los dispositivos debe ser de forma inalámbrica.
		\item Dispositivos de medición de caudal de agua deberán usar tecnología ultrasónica
		\item Solución debe ser atender primariamente las salidas de agua de servicios higiénicos y/o de tuberías de hasta 1".
		\item La transmisión de datos debe ser ligera y debe consumir baja energía. 
		\item Solución debe contar con documentación de cobertura de señal. 
		\item Solución debe incluir documentación con estudio de parámetros para medición de caudal de agua usando sensores ultrasónicos.
		\item Software embebido debe poder encapsularse en un contenedor docker.
		\item Aplicativo web debe permitir registro de ubicación del sensor y asociarla con la colección de datos.
		\item Aplicativo web debe mostrar caudal de agua de cada sensor en tiempo real. 
		\item Aplicativo web debe mostrar caudal histórico de cada sensor y promedio de los últimos 7 días.
		\item Solución debe incluir notificaciones que permitan alertar a usuarios de alteración del caudal habitual, superior al 10\% al promedio histórico.		
		\item Opcionalmente, solución debe tener capacidad portátil para conexión de datos, alimentación eléctrica e instalación de plomería.
	\end{enumerate}

\item Requerimientos No Funcionales
	\begin{enumerate}
		\item Adecuación a lineamientos establecidos en la ISO 30141:2018.
		\item Los sensores deberán conectarse a una red de forma predeterminada.
		\item Activación de configuraciones de seguridad disponibles en los componentes de la solución.
		\item Opcionalmente, solución podrá tener control de accesos configurable.
	\end{enumerate}

\end{enumerate}

\end{consigna}

\section{7. Historias de usuarios (\textit{Product backlog})}
\label{sec:backlog}

\begin{consigna}{black}
A continuación se listas las historias de usuario y su ponderación, que es un número entero que representa el tamaño de la historia comparada con otras historias de similar tipo. Está basada en la serie de Fibonacci.

Los criterios para la ponderación son los siguientes:

\begin{table}[ht]
\label{tab:haUsuarioPesos}
\centering
\begin{tabularx}{\linewidth}{@{}|c|X|c|@{}}
\hline
\rowcolor[HTML]{C0C0C0} 
\# & \multicolumn{1}{c|}{\cellcolor[HTML]{C0C0C0}Criterio} 	& Peso      \\ \hline
1      & \textbf{Dificultad de trabajo a realizar}          	&  \\ \hline 
      	& Bajo                 								&  1 \\ \hline
       	& Medio                								&  3 \\ \hline
      	& Alto                 								&  5 \\ \hline
2      & \textbf{Complejidad de trabajo a realizar}        	&  \\ \hline 
      	& Bajo                 								&  1 \\ \hline
       	& Medio                								&  5 \\ \hline
      	& Alto                 								& 13 \\ \hline
2      & \textbf{Riesgo-incertidumbre de trabajo a realizar}        	&  \\ \hline 
      	& Bajo                 								&  1 \\ \hline
       	& Medio                								&  5 \\ \hline
      	& Alto                 								& 13 \\ \hline

\end{tabularx}
\end{table}

Las historias de usuario son las siguientes:

\begin{enumerate}
\item Como cliente quiero que la medición del caudal sea lo más precisa posible, para tener confianza en la solución y compararla con mediciones de la empresa de suministro de agua
	\begin{itemize}
		\item Dificultad	:  5
		\item Complejidad	:  8
		\item Riesgo		: 13
		\item Total			: 26
		\item \textbf{\textit{Story Points	:} 34} 		
	\end{itemize}

\item Como cliente quiero que la aplicación web sea responsiva, para que sea usada desde computadoras y móviles
	\begin{itemize}
		\item Dificultad	:  3
		\item Complejidad	:  8
		\item Riesgo		:  5
		\item Total			: 16
		\item \textbf{\textit{Story Points	:} 21}		
	\end{itemize}

\item Como cliente deseo que la solución proporcione información histórica por ubicación del sensor, para comparación con consumos pasados
	\begin{itemize}
		\item Dificultad	:  4
		\item Complejidad	: 10
		\item Riesgo		:  7
		\item Total			: 21
		\item \textbf{\textit{Story Points	:} 21}		
	\end{itemize}

\item Como cliente deseo que la solución notifique ante alteración de patrón de consumo desviado en un 10\% mayor con respecto al consumo histórico, para tener una alerta oportuna de posibles fugas.
	\begin{itemize}
		\item Dificultad	:  5
		\item Complejidad	: 15
		\item Riesgo		: 10
		\item Total			: 30
		\item \textbf{\textit{Story Points	:} 34}		
	\end{itemize}
	
\item Como cliente deseo que la solución sea portátil, para poder moverla en distintos ambientes 
	\begin{itemize}
		\item Dificultad	:  4
		\item Complejidad	:  5
		\item Riesgo		: 13
		\item Total			: 24
		\item \textbf{\textit{Story Points	:} 34}		
	\end{itemize}
\end{enumerate}

\end{consigna}

\section{8. Entregables principales del proyecto}
\label{sec:entregables}

\begin{consigna}{black}

Los entregables del proyecto son:

\begin{itemize}
	\item Prototipo final de solución
	\item Aplicación web implementada
	\item Documentación del proyecto 
	\begin{itemize}
		\item Diagrama de arquitectura de solución
		\item Diagrama de circuitos esquemáticos
		\item Código fuente del software
		\item Diagrama de instalación
		\item Memoria técnica final
	\end{itemize}
	
\end{itemize}

\end{consigna}

\section{9. Desglose del trabajo en tareas}
\label{sec:wbs}

\begin{consigna}{black}

\begin{enumerate}
\item \textbf{Planificación del proyecto: 40 hs}
	\begin{enumerate}
	\item Elaboración del plan de proyecto (20 hs)
	\item Análisis de factibilidad (10 hs)
	\item Validaciones de calidad (10 hs)
	\end{enumerate}
\item \textbf{Investigación inicial: 60 hs}
	\begin{enumerate}
	\item Investigación sobre sensores de medición de flujo de agua con tecnología ultrasónica (30 hs)
	\item Indagación sobre nivel de programación de sensores identificados (20 hs)
	\item Proceso de adquisición de sensores (10 hs)
	\end{enumerate}
\item \textbf{Diseño de arquitectura de solución: 60 hs}
	\begin{enumerate}
	\item Arquitectura tecnológica (15 hs)
	\item Arquitectura de aplicación (20 hs)
	\item Arquitectura de integración (10 hs)
	\item Arquitectura de datos (15 hs)
	\end{enumerate}
\item \textbf{Desarrollo de aplicación web: 150 hs}
	\begin{enumerate}
	\item Desarrollo front end (50 hs)
	\item Desarrollo back end (60 hs)
	\item Pruebas de aplicaciones (30 hs)
	\item Habilitación de infraestructura (10 hs)
	\end{enumerate}
\item \textbf{Desarrollo de sistema embebido: 180 hs}
	\begin{enumerate}
	\item Desarrollo de algoritmo de medición de caudal de agua (50 hs)
	\item Desarrollo de algoritmo de comunicaciones (40 hs)
	\item Desarrollo de algoritmo de manejo de datos (40 hs)
	\item Encapsulamiento de aplicación en contenedores (20 hs)
	\item Corrección de errores (30 hr)
	\end{enumerate}
\item \textbf{Elaboración de documentación: 120 hs}
	\begin{enumerate}
	\item Confección de memoria técnica (60 hs)
	\item Informes de avance (20 hs)
	\item Presentación de defensa (20 hs)
	\item Encapsulamiento de aplicación en contenedores (20 hs)
	\end{enumerate}
\end{enumerate}

\textbf{Cantidad total de horas: 610 hs}

\end{consigna}

\section{10. Diagrama de Activity On Node}
\label{sec:AoN}

\begin{consigna}{red}
Armar el AoN a partir del WBS definido en la etapa anterior. 

%La figura \ref{fig:AoN} fue elaborada con el paquete latex tikz y pueden consultar la siguiente referencia \textit{online}:

%\url{https://www.overleaf.com/learn/latex/LaTeX_Graphics_using_TikZ:_A_Tutorial_for_Beginners_(Part_3)\%E2\%80\%94Creating_Flowcharts}

\end{consigna}

\begin{figure}[htpb]
\centering 
\includegraphics[width=.8\textwidth]{./Figuras/AoN.png}
\caption{Diagrama en \textit{Activity on Node}}
\label{fig:AoN}
\end{figure}

Indicar claramente en qué unidades están expresados los tiempos.
De ser necesario indicar los caminos semicríticos y analizar sus tiempos mediante un cuadro.
Es recomendable usar colores y un cuadro indicativo describiendo qué representa cada color, como se muestra en el siguiente ejemplo:



\section{11. Diagrama de Gantt}
\label{sec:gantt}

\begin{consigna}{red}

Existen muchos programas y recursos \textit{online} para hacer diagramas de gantt, entre los cuales destacamos:

\begin{itemize}
\item Planner
\item GanttProject
\item Trello + \textit{plugins}. En el siguiente link hay un tutorial oficial: \\ \url{https://blog.trello.com/es/diagrama-de-gantt-de-un-proyecto}
\item Creately, herramienta online colaborativa. \\\url{https://creately.com/diagram/example/ieb3p3ml/LaTeX}
\item Se puede hacer en latex con el paquete \textit{pgfgantt}\\ \url{http://ctan.dcc.uchile.cl/graphics/pgf/contrib/pgfgantt/pgfgantt.pdf}
\end{itemize}

Pegar acá una captura de pantalla del diagrama de Gantt, cuidando que la letra sea suficientemente grande como para ser legible. 
Si el diagrama queda demasiado ancho, se puede pegar primero la ``tabla'' del Gantt y luego pegar la parte del diagrama de barras del diagrama de Gantt.

Configurar el software para que en la parte de la tabla muestre los códigos del EDT (WBS).\\
Configurar el software para que al lado de cada barra muestre el nombre de cada tarea.\\
Revisar que la fecha de finalización coincida con lo indicado en el Acta Constitutiva.

En la figura \ref{fig:gantt}, se muestra un ejemplo de diagrama de gantt realizado con el paquete de \textit{pgfgantt}. En la plantilla pueden ver el código que lo genera y usarlo de base para construir el propio.

\begin{figure}[htbp]
\begin{center}
\begin{ganttchart}{1}{12}
  \gantttitle{2020}{12} \\
  \gantttitlelist{1,...,12}{1} \\
  \ganttgroup{Group 1}{1}{7} \\
  \ganttbar{Task 1}{1}{2} \\
  \ganttlinkedbar{Task 2}{3}{7} \ganttnewline
  \ganttmilestone{Milestone o hito}{7} \ganttnewline
  \ganttbar{Final Task}{8}{12}
  \ganttlink{elem2}{elem3}
  \ganttlink{elem3}{elem4}
\end{ganttchart}
\end{center}
\caption{Diagrama de gantt de ejemplo}
\label{fig:gantt}
\end{figure}


\begin{landscape}
\begin{figure}[htpb]
\centering 
\includegraphics[height=.85\textheight]{./Figuras/Gantt-2.png}
\caption{Ejemplo de diagrama de Gantt rotado}
\label{fig:diagGantt}
\end{figure}

\end{landscape}

\end{consigna}


\section{12. Presupuesto detallado del proyecto}
\label{sec:presupuesto}

\begin{consigna}{red}
Si el proyecto es complejo entonces separarlo en partes:
\begin{itemize}
	\item Un total global, indicando el subtotal acumulado por cada una de las áreas.
	\item El desglose detallado del subtotal de cada una de las áreas.
\end{itemize}

IMPORTANTE: No olvidarse de considerar los COSTOS INDIRECTOS.

\end{consigna}

\begin{table}[htpb]
\centering
\begin{tabularx}{\linewidth}{@{}|X|c|r|r|@{}}
\hline
\rowcolor[HTML]{C0C0C0} 
\multicolumn{4}{|c|}{\cellcolor[HTML]{C0C0C0}COSTOS DIRECTOS} \\ \hline
\rowcolor[HTML]{C0C0C0} 
Descripción &
  \multicolumn{1}{c|}{\cellcolor[HTML]{C0C0C0}Cantidad} &
  \multicolumn{1}{c|}{\cellcolor[HTML]{C0C0C0}Valor unitario} &
  \multicolumn{1}{c|}{\cellcolor[HTML]{C0C0C0}Valor total} \\ \hline
 &
  \multicolumn{1}{c|}{} &
  \multicolumn{1}{c|}{} &
  \multicolumn{1}{c|}{} \\ \hline
 &
  \multicolumn{1}{c|}{} &
  \multicolumn{1}{c|}{} &
  \multicolumn{1}{c|}{} \\ \hline
\multicolumn{1}{|l|}{} &
   &
   &
   \\ \hline
\multicolumn{1}{|l|}{} &
   &
   &
   \\ \hline
\multicolumn{3}{|c|}{SUBTOTAL} &
  \multicolumn{1}{c|}{} \\ \hline
\rowcolor[HTML]{C0C0C0} 
\multicolumn{4}{|c|}{\cellcolor[HTML]{C0C0C0}COSTOS INDIRECTOS} \\ \hline
\rowcolor[HTML]{C0C0C0} 
Descripción &
  \multicolumn{1}{c|}{\cellcolor[HTML]{C0C0C0}Cantidad} &
  \multicolumn{1}{c|}{\cellcolor[HTML]{C0C0C0}Valor unitario} &
  \multicolumn{1}{c|}{\cellcolor[HTML]{C0C0C0}Valor total} \\ \hline
\multicolumn{1}{|l|}{} &
   &
   &
   \\ \hline
\multicolumn{1}{|l|}{} &
   &
   &
   \\ \hline
\multicolumn{1}{|l|}{} &
   &
   &
   \\ \hline
\multicolumn{3}{|c|}{SUBTOTAL} &
  \multicolumn{1}{c|}{} \\ \hline
\rowcolor[HTML]{C0C0C0}
\multicolumn{3}{|c|}{TOTAL} &
   \\ \hline
\end{tabularx}%
\end{table}


\section{13. Gestión de riesgos}
\label{sec:riesgos}

\begin{consigna}{red}
a) Identificación de los riesgos (al menos cinco) y estimación de sus consecuencias:
 
Riesgo 1: detallar el riesgo (riesgo es algo que si ocurre altera los planes previstos de forma negativa)
\begin{itemize}
	\item Severidad (S): mientras más severo, más alto es el número (usar números del 1 al 10).\\
	Justificar el motivo por el cual se asigna determinado número de severidad (S).
	\item Probabilidad de ocurrencia (O): mientras más probable, más alto es el número (usar del 1 al 10).\\
	Justificar el motivo por el cual se asigna determinado número de (O). 
\end{itemize}   

Riesgo 2:
\begin{itemize}
	\item Severidad (S): 
	\item Ocurrencia (O):
\end{itemize}

Riesgo 3:
\begin{itemize}
	\item Severidad (S): 
	\item Ocurrencia (O):
\end{itemize}


b) Tabla de gestión de riesgos:      (El RPN se calcula como RPN=SxO)

\begin{table}[htpb]
\centering
\begin{tabularx}{\linewidth}{@{}|X|c|c|c|c|c|c|@{}}
\hline
\rowcolor[HTML]{C0C0C0} 
Riesgo & S & O & RPN & S* & O* & RPN* \\ \hline
       &   &   &     &    &    &      \\ \hline
       &   &   &     &    &    &      \\ \hline
       &   &   &     &    &    &      \\ \hline
       &   &   &     &    &    &      \\ \hline
       &   &   &     &    &    &      \\ \hline
\end{tabularx}%
\end{table}

Criterio adoptado: 
Se tomarán medidas de mitigación en los riesgos cuyos números de RPN sean mayores a...

Nota: los valores marcados con (*) en la tabla corresponden luego de haber aplicado la mitigación.

c) Plan de mitigación de los riesgos que originalmente excedían el RPN máximo establecido:
 
Riesgo 1: plan de mitigación (si por el RPN fuera necesario elaborar un plan de mitigación).
  Nueva asignación de S y O, con su respectiva justificación:
  - Severidad (S): mientras más severo, más alto es el número (usar números del 1 al 10).
          Justificar el motivo por el cual se asigna determinado número de severidad (S).
  - Probabilidad de ocurrencia (O): mientras más probable, más alto es el número (usar del 1 al 10).
          Justificar el motivo por el cual se asigna determinado número de (O).

Riesgo 2: plan de mitigación (si por el RPN fuera necesario elaborar un plan de mitigación).
 
Riesgo 3: plan de mitigación (si por el RPN fuera necesario elaborar un plan de mitigación).

\end{consigna}


\section{14. Gestión de la calidad}
\label{sec:calidad}

\begin{consigna}{red}
Para cada uno de los requerimientos del proyecto indique:
\begin{itemize} 
\item Req \#1: copiar acá el requerimiento.

\begin{itemize}
	\item Verificación para confirmar si se cumplió con lo requerido antes de mostrar el sistema al cliente. Detallar 
	\item Validación con el cliente para confirmar que está de acuerdo en que se cumplió con lo requerido. Detallar  
\end{itemize}

\end{itemize}

Tener en cuenta que en este contexto se pueden mencionar simulaciones, cálculos, revisión de hojas de datos, consulta con expertos, mediciones, etc.  Las acciones de verificación suelen considerar al entregable como ``caja blanca'', es decir se conoce en profundidad su funcionamiento interno.  En cambio, las acciones de validación suelen considerar al entregable como ``caja negra'', es decir, que no se conocen los detalles de su funcionamiento interno.

\end{consigna}

\section{15. Procesos de cierre}    
\label{sec:cierre}

\begin{consigna}{red}
Establecer las pautas de trabajo para realizar una reunión final de evaluación del proyecto, tal que contemple las siguientes actividades:

\begin{itemize}
	\item Pautas de trabajo que se seguirán para analizar si se respetó el Plan de Proyecto original:
	 - Indicar quién se ocupará de hacer esto y cuál será el procedimiento a aplicar. 
	\item Identificación de las técnicas y procedimientos útiles e inútiles que se emplearon, y los problemas que surgieron y cómo se solucionaron:
	 - Indicar quién se ocupará de hacer esto y cuál será el procedimiento para dejar registro.
	\item Indicar quién organizará el acto de agradecimiento a todos los interesados, y en especial al equipo de trabajo y colaboradores:
	  - Indicar esto y quién financiará los gastos correspondientes.
\end{itemize}

\end{consigna}


\end{document}
